\documentclass[a4paper,11pt]{article}
%\documentclass[11pt,twocolumn]{article}
\usepackage[utf8]{inputenc}
\usepackage{graphicx}
\usepackage[spanish]{babel}
\usepackage{amsmath}
\usepackage{authblk}
\begin{document}
\author{
Aveiga Iván \\
\texttt{ijaveiga at gmail dot com}
}
\title{Influencia de los planetas en el hombre, astrología descartada}
\maketitle
\begin{abstract}
Uno de los m\'as grandes timos de la historia es la astrolog\'ia, la cual se mantiene en vigencia y genera
grandes sumas de dinero. Muchas personas dejan en poder de lo que dicta la astrolog\'ia el curso de su vida.
Uno de los argumentos de la \emph{Astrolog\'ia} es que los cuerpos celestes gobiernan la vida de los individuos, siendo
esto ya predeterminado en el nacimiento, veremos de una manera m\'as rigurosa que dicha influencia de los 
cuerpos es insignificante respecto a otros cuerpos.
%\keyword[Gravitaci\'on Ley,Constante gravitacional]
\end{abstract}
\newpage
\section{Introducción}
\emph{\textbf{Astrolog\'ia:}} palabra que significa \emph{estudio de los astros}, proveniente de dos vocablos griegos:
(\'astron): ' estrella' y (logos): 'tratado, estudio'. Trata del estudio de como los cuerpos celestes rigen el destino
de los seres humanos desde su nacimiento, y como saber el destino consultando a los cuerpos celestes.
En la antigüedad en muchos pa\'ises exist\'ia el \'astrologo oficia1$[1]$, era el encargado de leer los cuerpos celestes en el firmamento y nadie m\'as que ellos pod\'ia realizar observaciones y predicciones ya que pod\'ia alterar las decisiones del reino y llevarlos a la ruina.\\*
Muchos reyes y emperadores no empezaban sus labores cotidianas a menos que leyeran su carta astral, en muchos casos
fen\'omenos pocos observados como alg\'un cometa desconocido en el firmamento era una señal de que una tragedia pod\'ia 
amenazar al reino, incluso cabe recalcar que la palabra \emph{\textbf{tragedia}} proviene del italiano \emph{disastro} que significa
sin estrella.  \\ \\*
Una de las formas de saber el destino seg\'un la astrolog\'ia es por medio del hor\'oscopo, el signo del individuo  depende de la posici\'on de los astros al nacer, dichos astros y dichas posiciones iniciales fueron tomadas arbitrariamente en el inicio, dando as\'i a la \emph{creaci\'on} de doce signos zodiacales.

\begin{itemize}
\item{Aries}
\item{Tauro}
\item{G\'eminis}
\item{C\'ancer}
\item{Leo}
\item{Virgo}
\item{Libra}
\item{Escorpio}
\item{Sagitario}
\item{Capricornio}
\item{Acuario}
\item{Piscis}
\end{itemize}

Aunque en la actualidad la influencia de la astrolog\'ia ha disminuido al punto de que decisiones de los pa\'ises no se basaban
en ella, ha habido excepciones en que grandes l\'ideres se basaban en el hor\'oscopo para la toma de decisiones.
Entre l\'ideres relativamente actuales tenemos a Adolf Hitler[2], Stalin, Napole\'on, Ronald Reagan y Manuel Noriega (panameño)[3].
\\ \\*
Se determinar\'a de manera m\'as rigurosa que una persona ubicada a un metro de distancia de un reci\'en nacido ejerce m\'as influencia que un planeta sobre el reci\'en nacido, descartando as\'i la \emph{absurda} creencia de que el curso de la vida de los seres humanos queda determinado por los astros y su influencia.

 \section{Influencia planetaria desde la f\'isica}
Lo que me llev\'o a escribir este art\'iculo es formalizar un poco m\'as en t\'erminos cient\'ificos la frase del gran Maestro Carl Sagan, \quote{\emph{C\'omo puede la ascendencia de Marte en el momento de mi nacimiento influir sobre m\ií, ni entonces ni ahora. Yo nac\'i en una habitaci\'on cerrada, la luz de Marte no pod\'ia entrar. La \'unica influencia de Marte que pod\'ia afectarme era su gravitaci\'on, sin embargo la influencia gravitatoria del toc\'ologo era mucho mayor que la influencia gravitatoria de Marte. Marte tiene mayor masa, pero el toc\'ologo estaba mucho m\'as cerca}}
\\ \\*
Usando la f\'ormula vista en mec\'anica cl\'asica para determinar la fuerza gravitatoria entre dos cuerpos:
\begin{equation}
F = G\frac{m_1 m_2}{r^2}
\end{equation} 
Podemos medir la interacci\'on que existe entre un reci\'en nacido y el toc\'ologo, y,  entre el reci\'en nacido y alg\'un cuerpo celeste.
\\ \\* \emph{\textbf{Datos}}
	\begin{itemize}
	\item{$m_1$ (Masa adulto promedio de 1.65 m de estatura):  63 kg}
	\item{$m_2$ (Masa niño reci\'en nacido): 0.8 kg }
	\item{$m_3$ (Masa de Marte): $6.4 x 10^24kg$}
	\item{$r_{12}$ (Distancia de $m_1$ a $ m_2$): 1m}
	\item{$r_{13}$ (Distancia de $m_1$ a $ m_3$): $56 X 10^{12} m$}
	\item{$G$ (Cte gravitacional): $6.7 x 10^{-11}  \frac{m^3}{kg  s^2}$}
	\end{itemize}

\begin{equation}
F_{12} = G\frac{m_1 m_2}{r_{12}^2}
\end{equation}
\begin{equation}
F_{12} = 6.7x10^{-11}\frac{m^3}{kg s^2} \frac{(63 kg) (0.8kg)}{(1m)^2}
\end{equation}
\begin{equation}
F_{12} = 3.376x10^{-8} N
\end{equation}
\begin{equation}
F_{13} = G\frac{m_1 m_3}{r_{13}^2}
\end{equation}
\begin{equation}
F_{13} = 6.7x10^{-11}\frac{m^3}{kg s^2} \frac{(6.4 x 10^{24} kg) (0.8kg)}{(56 X 10^{12}m)^2}
\end{equation}
\begin{equation}
F_{13} = 1.09x10^{-13} N
\end{equation}

Podemos observar que $F_{12}$ que corresponde a la fuerza de interacci\'on entre el reci\'en nacido y el adulto promedio es
mucho m\'as grande que $F_{13}$ que corresponde a la fuerza de interaci\'on entre el reci\'en nacido y Marte.
\footnote{La distancia entre la Tierra y Marte fue la correspondiente a agosto 27, 2003 una de las m\'as cercanas a la Tierra.}\\* \\*
Obteniendo el cociente entre $\frac{F_{12}}{F_{13}} = 308700$, podemos ver que $F_{12}$ es 308700 veces m\'as fuerte 
que $F_{13}$

\section{Conclusiones}
Se puede demostrar cient\'ificamente un caso m\'as de invalidez de la astrolog\'ia, una forma de engaño que a\'un tiene influencia en las personas y dar a conocer a las personas m\'etodos formales que la ciencia provee para desterrar a la pseudociencia. 
Se debe crear un escepticismo entre las personas y este tipo de pseudociencias, que lo \'unico que buscan es el beneficio econ\'omico o de poder de algunos.
\begin{thebibliography}{2}
 \bibitem{CS88} C. Sagan , \emph{Cosmos, un viaje personal},1988
\bibitem{AR} A. Robert, \emph{Los arcanos negros de Hitler},2005
\bibitem{MJ} J. Monge-N\'ajera \emph{El ser humano en su entorno},2007
\end{thebibliography}
\end{document}


