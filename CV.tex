%%%%%%%%%%%%%%%%%%%%%%%%%%%%%%%%%%%%%%%%%%
%%	Curriculum Vitae					%%
%%	Lenguajes de Programación			%%
%%	II Término 2012 					%%
%%%%%%%%%%%%%%%%%%%%%%%%%%%%%%%%%%%%%%%%%%

%%%%%%%%%%%%%%%%%%%%%%%%%%
%Author:	Iván Aveiga %%
%Date:		8/10/2012	%%
%%%%%%%%%%%%%%%%%%%%%%%%%%

%Paquetes y declaración de tipo de documento
\documentclass[a4paper,11pt]{article}
\usepackage{graphicx}
\usepackage[utf8]{inputenc}
\usepackage[activeacute,spanish]{babel}
\usepackage{authblk}
\usepackage[margin=3cm]{geometry}
\usepackage{fancyhdr}
\pagestyle{fancyplain}
\title{\bf{Curriculum Vitae}}
\author{ijaveiga@gmail.com}
\begin{document}
\fancyhead[R]{\emph{Curriculum Vitae - Lenguajes de Programación 2012 II}}
\fancyfoot[R]{\emph{Iván Aveiga}}
\maketitle
\begin{figure}[h]
\centering
\includegraphics[width=0.6\linewidth]{logo}
\end{figure}
\lhead{}
\newpage
\section{\textbf{Datos Personales}}
\noindent
\textbf{Nombres:} \hspace{59mm}  Iván José \\ 
\textbf{Apellidos:} \hspace{58mm}  Aveiga Adanaqué \\ 
\textbf{Fecha de Nacimiento:} \hspace{36mm}  30 - 07 - 1990 \\ 
\textbf{Ciudad:} \hspace{62mm} Guayaquil \\ 

\section{\textbf{Estudios en Curso}}
\noindent
Curso un Bachelor Sciences en Ciencias de la Computación, especialidad Sistemas Tecnológicos en la
Escuela Superior Politécnica del Litoral.

\section{\textbf{Áreas de Interés:}}
\noindent
A continuación presento tanto áreas de interés educativo y profesional como personal.
	\subsection{Educación y Profesional}
		\begin{itemize}
			\item \textbf{Objetos de Aprendizaje:} Aprovechar estas tecnologías para poder lograr software educativo de
					calidad tanto en contenido como en presentación.
			\item \textbf{Neurociencia Computacional:} Una gran área de interés donde se puede aprovechar la forma en  	cómo se realizan los procesos de pensamiento (sinápis, otros) y aplicarlos a modelos computacionales.
			\item {\textbf{Python:}} Aprovechar las ventajas de \emph{Python} gracias a las librerias que son desarrolladas por comunidades de software libre, entre las más importantes con las cuáles se puede realizar proyectos muy interesantes están:
				\begin{itemize}
					\item \emph{Scipy:} Cálculo Científico.
					\item \emph{Numpy:} Cálculo Numérico.
					\item \emph{Matplotlib:} Ploteo de gráficos.
					\item \emph{Pybrain:} Machine Learning.
					\item \emph{PIL:} Libreria de procesamiento de imágenes.
				\end{itemize}
		\end{itemize}
	\subsection{Personal} 
		\begin{itemize}	
			\item{\emph{Divulgación Científica}}
			\item{\emph{Escritura de artículos en \LaTeX de divulgación}}
			\item{\emph{Recopilación de ebooks sobre desarrollo de software}}
		\end{itemize}
\newpage
\section{Certificaciones y Cursos Realizados:}		
\noindent
	\begin{itemize}
		\item \textbf{Database Administration Fundamentals} \\
		\emph{Microsoft}
		
		\item \textbf{Software Development Fundamentals} \\
		\emph{Microsoft}
		
		\item \textbf{Introducción a Visual C++} \\
		\emph{Academia Microsoft ESPOL}
		
		\item \textbf{Introduction to Computer Sciences 101} \\
		\emph{Coursera}
	\end{itemize}
\end{document}
